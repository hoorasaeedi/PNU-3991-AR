\documentclass{book}
\begin{document}
\begin{flushright}
\texttt{INTRODUCTION}
\hspace*{0.5cm}
\textbf{86}
\end{flushright}
\vspace*{0.7cm}

\textbf{CHAPTER SEVEN} \\

group (Borg & Gall, 1989; Cohen & Manion, 1994). In a structured interviews the
researcher asks the same sequenced and preestablished questions to a large number of
respondents in a standardized manner, often with response categories. Typically,
there is no option for variations in response. While a structured interview, generally
referred to as a survey, can be conducted one-on-one with the researcher, it is most
commonly presented in a paper-based or online format to a large sample (see Chapter
11). The data form structured interviews are usually analyzed numerically using
descriptive statistics such as means and standard deviations or multivariate statistics
such as cluster and factor analysis, with the reliability of internal scales being calculated
using sophisticated mathematical formulas such as Coronach's alpha. Since these types
of interviews are presented to large and carefully selected samples, the results are often
evaluated in terms of their generalizability. Conversely, focus group interviews (see
Chapter 8) are never conducted one-on-one or with a large sample; rather, they are
conducted as a small group interview. The data are analyzed for themes and topics, and
the researcher is usually not concerned with the generalizability of results. Rather, the
data is evaluated for its ability to provide insights into the issue(s) being inves-
tigated. Alternatively, semi-structured and unstructured interviews are conducted one-
on-one. While the most typical setting for semi-structured and unstructured
interviews continues to be a face-to-face verbal interchange, there is an increasing
opportunity for individual interviews to be conducted on the Net. \\
Whatever the format, interviews are a favorite methodological tool of educa-
tional and social science researchers. According to Fontana and Frey (1994), "inter-
viewing is one of the most common and most powerful ways we use to try to
understand our fellow human beings" (p. 361). Although, Patton (1990) suggests that
the "quality of the information obtained during an interview is largely dependent
on the interviewer" (p.279). To obtain quality information, the e-researcher needs to
not only be a skilled interviewer, but must also be able to transfer these skills to the Net
environment. This chapter focuses on the interviewing skills and questioning tech-
niques necessary to create engaging conversations for effective Net-based semi-
structured and unstructured interviews. \\
\\
\vspace*{0.3cm}
\\
\textbf{UNSTRUCTURED VERSUS SEMI-STRUCTURED INTERVIEWS}
\\
\vspace*{0.3cm}
\\
Unstructured interviews are often referred to as in-depth interviews, open-ended
interviews, or even ethnographic research. While we acknowledge that many qualita-
tive researchers do not differentiate between ethnographic research and unstructured
interviews (e.g.,Fontana & Frey, 1994; Lofland, 1971), we do make this distinction.
The frequently cited writing of Lofland maintains, for example, that in-depth inter-
views and ethnographic research go hand-in-hand. However, Net-based in-depth
interviews typically do not use participant observation in natural settings, and in this
context are considered to be fundamentally different form ethnographic research. As
Internet technologies advance, in terms of cost and ease of use, we anticipate that in the
not-too-distant future Net-based interviews will take place in virtual reality environ-
ments and make use of voice and video interaction. They will then allow e-researchers
\end{document}
