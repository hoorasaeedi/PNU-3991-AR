\documentclass{book}
\begin{document}
\begin{flushright}
\texttt{INTRODUCTION}
\hspace*{0.5cm}
\textbf{87}
\end{flushright}
\vspace*{0.7cm}q

SEMI-STRUCTURED AND UNSTRUCTURED INTERVIEWS
\\
\vspace*{0.1cm}
\\
to conduct interviews and observe participants in their natural setting, thus making
Net-based ethnographic research more commonplace. While this interview process is
possible now, it is currently beyond the means of most researchers and their subjects
in terms of the cost, required skills, and equipment.
An overview of the different kinds of Internet communication software is pro-
vided in Chapter 8(Focus Groups). There are researchers who conduct ethnographic
research in a Net-based context-where the objective of the study is to explore the
social culture on the Net as well as the use of the Net for data collection. An interest-
ing example of data collection can be found in an early, groundbreaking ethnographic
research study about the psychology of online life, in Sherry Turkle's book, \emph{Life on the
Screen} (1995). For a variety of reasons (the primary one being financial), most
researchers are exploring text-based asynchronous formats for interviewing. As such,
this chapter focuses on text-based, asynchronous, and Net-based interviews. \\
Unstructured interviews provide the greatest potential for the researcher to
achieve breadth and depth of data. This type of interview includes a small number of
loosely defined questions (sometimes only one question) that provide openings for the
participants to describe their views in their own language and style. There are no pre-
determined questions asked of each participant and no precise order of the questions;
the interviewer converses with the respondent as questions arise. Usually questions are
broad and open-ended. thus providing maximum opportunity for the participant to
shape answers in a meaningful way. In turn, the interviewer clarifies the responses for
deeper insights through paraphrasing, reflective comments, and follow-up questions
(Snyder, 1992). Rapport between researcher and respondent and an understanding of
respondents' experiences take precedence over data formatted into preestablished,
coded categories. \\
Semi-structured interviews are conducted with specific topics in mind,
form which questions are generated based on a theoretical framework. Typically, the
interviewer works form an interviews schedule that contains a series of preplanned and
sequenced questions. These questions may be followed by less structured and open-
ended probes (follow-up questions) to collect deeper understandings and insights. The
main advantage of a semi-structured interview over an unstructured interview is that in
the former there is both structure (ordered questions) and non-structure (open-ended
probes). Thus the interviewer can both predetermine the data that will be gathered (in
a structured interview) and follow the unexpected as it arises (in an unstructured
interview).
\\
\vspace*{0.3cm}
\\
\textbf{INTERVIEWING SKILLS}
\\
\vspace*{0.3cm}
\\
Regardless of the type of interview format, forethought concerning the form, struc-
ture, and purpose of the interview is essential when using this method for data collec-
tion. An interview can take a variety of forms for a multiplicity of purposes. It can be a
quick, one-time, five-minute exchange or multiple interactions extending over a num-
ber of days, weeks, months, or even years. Whatever the format, there are a number of
steps the r-researcher should take prior to during the interview. Some of these
\end{document}
