\documentclass{book}
\begin{document}
\begin{flushright}
\texttt{INTRODUCTION}
\hspace*{0.5cm}
\textbf{85}
\end{flushright}
\vspace*{0.7cm}
\textbf{CHAPTER SEVEN} \\
\textbf{SEMI-STRUCTURED AND \\
UNSTRUCTURED INTERVIEWS} \\
\\
\vspace*{1cm}
\\
\emph{Some things cannot be spoken or discovered until we have been stuck,
incapacitated, or blown off course for a while. plain sailing is pleasant, but you
are not going to explore many unknown realms that way.} \\
David Whyte \\
\\
\vspace*{1cm}
\\
The research method e-researchers choose can often reveal their beliefs about what is
valuable knowledge and, perhaps more importantly, perspectives of the nature of
reality. As discussed in Chapter 3, an objective of qualitative research methods is the
discovery of patterns and the development of theories that expand understanding of
complex social phenomena. When conducting qualitative research the researcher usu-
ally uses an in-depth inductive process. Semi-structured and unstructured interviews
are the most common methods for achieving this deep understanding of complex social
phenomena. The interview is a unique method for data collection in that the
researcher gathers data through direct communication between individuals. This direct
communication allows for customization of the questions, depending on the subjects'
previous answers, their attitudes, and the trust that builds between the researcher and
the participants. Such direct and focused interaction is necessary to pursue a deep
understanding of the subjects' views and of the topic of investigation. In particular,
interviews offer the researcher infinite flexibility in probing deeper by following leads
and insights that may provide surprising new directions for both participants and
researcher. This is the principle advantage of interviews over other kinds of research
methods. \\
There are many different kinds of interviews used by e-researchers. In fact, there
is considerable variation within the literature on types and definitions of interview
genres. Fontana and Frey (1994), for example, describe nine types of interviews includ-
ing structured, semi-structured, oral history, creative, group, postmodern, gendered,
ethnographic, and in-depth interviews. Most of the literature, however, lists the most
common kinds of interviews as structured, unstructured, and focus
\end{document}


